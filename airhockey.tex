%%%%%%%%%%%%%%%%%%%%%%%%%%%%%%%%%%%%%%%%%%%%%%%%%%%%%%%%%%%%%%%%
%%                     Semester project 2 	              %%
%%                   Table-less air hockey   		      %%
%%                       Andrew Watson                        %%
%%                           MT-MA1                           %%
%%%%%%%%%%%%%%%%%%%%%%%%%%%%%%%%%%%%%%%%%%%%%%%%%%%%%%%%%%%%%%%%

\input{includes/preamble}
% duplicating graphicx here because of vim-latex autodetect
\newcommand{\thedate}{\today}
\usepackage[pdftex]{graphicx} 
\usepackage{pdfpages}
\begin{document}
\begin{titlepage}
\nocite{*}      % to make sure bibliography appears in the correct order
  \begin{center}
     
     
    % Upper part of the page
    \includegraphics[width=4cm]{logo_epfl}\\[1.5cm]
     
    \textsc{\LARGE Microengineering }\\[1.0cm]

    \textsc{\Large Semester project at LAI}\\[0.1cm]

    \vfill 
     
    % Title
    \HRule \\[0.7cm]
    { \huge \bfseries Self-sustaining air hockey puck}\\[0.2cm]
    { \large \bfseries Design and study of an actuator}\\[0.4cm]

    \includegraphics[width=0.6\textwidth]{hockey-title} 
     
    \HRule \\[2.0cm]
    
    %% Author and supervisor
    \begin{minipage}{0.4\textwidth}
      \begin{flushleft} \large
        \emph{Author:} \\
        Andrew \textsc{Watson}\\[0.7cm]

        Microengineering\\
        Master - Semester 2\\[0.5cm]
      \end{flushleft}
    \end{minipage}
    \begin{minipage}{0.4\textwidth}
      \begin{flushright} \large
        \emph{Supervisor:} \\
        Yves \textsc{Perriard}\\[0.7cm]

        \emph{Assistant:} \\
	Christophe \textsc{Winter}\\[0.5cm]
      \end{flushright}
    \end{minipage} \\[2cm]
     
    \vfill
     
    % Bottom of the page
    {\large \thedate}
     
  \end{center}

\end{titlepage}
%\maketitle

\newpage{}

%\fancyhead{}
\fancyfoot{}
\lhead{}
\cfoot{\thepage}        % numéro de page..
\lfoot{Semester Project}
%\rfoot{\thedate}
\rfoot{\thedate} 

%\begin{abstract}
%\end{abstract}

\setcounter{secnumdepth}{5}

%\renewcommand\contentsname{Plan}  % Rename ``Table des Matières''
\tableofcontents{}

\newpage

%%%% BEGIN : LSTLISTINGS CONFIG %%%%%%
%%%% TODO : MOVE TO SEPARATE FILE ONCE FINISHED %%%%%
%% see http://www.jorgemarsal.com/blog/2009/06/08/source-code-snippets-in-latex/
\lstset{language=C}
%\definecolor{lightgrey}{RGB}{200,200,200}
\definecolor{grey97}{gray}{0.97}
\definecolor{grey92}{gray}{0.92}
\definecolor{grey75}{gray}{0.75}
\definecolor{grey45}{gray}{0.45}

\lstdefinestyle{console}
{
  numbers=none,
  %basicstyle=\bf\ttfamily,
  basicstyle=\ttfamily\footnotesize,
  backgroundcolor=\color{grey97},
  frame=lrtb,
  framerule=0.5pt,
  linewidth=\textwidth,
}
\lstdefinestyle{avr-c}
{
  style=console
}

\lstset{
  style=console
}

%%%%%%% END : LSTLISTINGS CONFIG %%%%%%%%


\section*{Introduction}
\addcontentsline{toc}{section}{Introduction}
\markboth{Introduction}{\MakeUppercase{Introduction}}

Currently, the game of air hockey is played with a puck and two 
% TODO: find name
paddles.
Both the paddles and the puck are passive pieces of plastic. The game therefore
necessitates a heavy and expensive air table to generate a thin film of air,
which reduces the friction between the puck and the table.

The premise of this project is to design an actuator similar to an air hockey
puck, which can generate its own air film and thus function on any flat surface.

Talk about:
Cost

Scope of project (i.e. not going to actually build one, calculate with
approximations, then simulate)

Clarify approach (steps)

\asterbreak


\section{Operating principle}
% TODO : explain principle

The self-levitating puck does not need an external air supply. Instead, it
creates its own air film by vibrating at a high frequency.

The puck is composed of four main elements:

% TODO : create picture with letters
\begin{figure}[h]
  \begin{center}
    \includegraphics[width=0.8\textwidth]{overview}
  \end{center}
  \caption{Overall view}
  \label{fig:overview}
\end{figure}

% TODO : convert to list with letters
% TODO : make this list more brief, try to make a mory lengthy, wordy and easy
% to follow explanation
\begin{description}
  \item[A] The original puck, made of yellow ABS plastic. This is the base
    around which the system is built. The main idea of this project is to re-use
    a standard puck, according to USAA rules. With the addition of the other
    components, it will no longer be acceptable by official game rules though.
  \item[B] The membrane is made of FR-4 (an epoxy mainly used in circuit board
    design). This membrane will resonate to create the pressure necessary to
    lift the contraption.
  \item[C] An aluminium support ring will provide mechanical robustness to the
    device, as well as provide one fixed point for the membrane to move around.
  \item[D] A piezo-electric actuator, in this case a ring stack, will actuate
    the membrane, exciting it at one of its resonant frequencies, in order to
    get the best possible vertical displacement.
\end{description}

% TODO : explain how the squeeze film works. Why there is more air coming in
% than going out. add graphs from Wiesendanger


\section{Conclusion}


The hardware design is functional, and can be demonstrated as requested.
However, not everything is fully functional, and many areas could be improved.
Some of these, mostly the hardware changes, are described in section
project, a lot of time was spent researching solutions. In addition to this, the
hardware design required more time than expected.

All these, I believe, could be worth exploring in the future, and I hope that
this platform may serve a purpose beyond the scope of this semester project.

\vspace{3cm}
Ecublens, le xx Juin 2011

\vspace{2cm}
Andrew Watson



\pagebreak
%%%%%%%%%%%%%%%%%%%%%%%%%%%%%%%%%%%%%%%%%%%%%%%%%%%%%%%%%%%%%%%%%%%%%%%%%%%%%%%
\bibliographystyle{ieeetr}
\bibliography{biblio-airhockey}

\pagebreak
%%%%%%%%%%%%%%%%%%%%%%%%%%%%%%%%%%%%%%%%%%%%%%%%%%%%%%%%%%%%%%%%%%%%%%%%%%%%%%%
\appendix

\section{Design changes}
\label{sec:design-changes}
\end{document}

