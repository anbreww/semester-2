%%%%%%%%%%%%%%%%%%%%%%%%%%%%%%%%%%%%%%%%%%%%%%%%%%%%%%%%%%%%%%%%
%%                     Semester project 2 	              %%
%%                   Table-less air hockey   		      %%
%%                       Andrew Watson                        %%
%%                           MT-MA1                           %%
%%%%%%%%%%%%%%%%%%%%%%%%%%%%%%%%%%%%%%%%%%%%%%%%%%%%%%%%%%%%%%%%

\input{includes/preamble}
% duplicating graphicx here because of vim-latex autodetect
\newcommand{\thedate}{\today}
\usepackage[pdftex]{graphicx} 
\usepackage{pdfpages}
\begin{document}
\begin{titlepage}
\nocite{*}      % to make sure bibliography appears in the correct order
  \begin{center}
     
     
    % Upper part of the page
    \includegraphics[width=4cm]{logo_epfl}\\[1.5cm]
     
    \textsc{\LARGE Microengineering }\\[1.0cm]

    \textsc{\Large Semester project at LAI}\\[0.1cm]

    \vfill 
     
    % Title
    \HRule \\[0.7cm]
    { \huge \bfseries Self-sustaining air hockey puck}\\[0.2cm]
    { \large \bfseries Design and study of an actuator}\\[0.4cm]

    \includegraphics[width=0.6\textwidth]{hockey-title} 
     
    \HRule \\[2.0cm]
    
    %% Author and supervisor
    \begin{minipage}{0.4\textwidth}
      \begin{flushleft} \large
        \emph{Author:} \\
        Andrew \textsc{Watson}\\[0.7cm]

        Microengineering\\
        Master - Semester 2\\[0.5cm]
      \end{flushleft}
    \end{minipage}
    \begin{minipage}{0.4\textwidth}
      \begin{flushright} \large
        \emph{Supervisor:} \\
        Yves \textsc{Perriard}\\[0.7cm]

        \emph{Assistant:} \\
	Christophe \textsc{Winter}\\[0.5cm]
      \end{flushright}
    \end{minipage} \\[2cm]
     
    \vfill
     
    % Bottom of the page
    {\large \thedate}
     
  \end{center}

\end{titlepage}
%\maketitle

\newpage{}

%\fancyhead{}
\fancyfoot{}
\lhead{}
\cfoot{\thepage}        % numéro de page..
\lfoot{Semester Project}
%\rfoot{\thedate}
\rfoot{\thedate} 

%\begin{abstract}
%\end{abstract}

\setcounter{secnumdepth}{5}

%\renewcommand\contentsname{Plan}  % Rename ``Table des Matières''
\tableofcontents{}

\newpage

%%%% BEGIN : LSTLISTINGS CONFIG %%%%%%
%%%% TODO : MOVE TO SEPARATE FILE ONCE FINISHED %%%%%
%% see http://www.jorgemarsal.com/blog/2009/06/08/source-code-snippets-in-latex/
\lstset{language=C}
%\definecolor{lightgrey}{RGB}{200,200,200}
\definecolor{grey97}{gray}{0.97}
\definecolor{grey92}{gray}{0.92}
\definecolor{grey75}{gray}{0.75}
\definecolor{grey45}{gray}{0.45}

\lstdefinestyle{console}
{
  numbers=none,
  %basicstyle=\bf\ttfamily,
  basicstyle=\ttfamily\footnotesize,
  backgroundcolor=\color{grey97},
  frame=lrtb,
  framerule=0.5pt,
  linewidth=\textwidth,
}
\lstdefinestyle{avr-c}
{
  style=console
}

\lstset{
  style=console
}

%%%%%%% END : LSTLISTINGS CONFIG %%%%%%%%


\section*{Introduction}
\addcontentsline{toc}{section}{Introduction}
\markboth{Introduction}{\MakeUppercase{Introduction}}

Currently, the game of air hockey is played with a puck and two 
% TODO: find name
paddles.
Both the paddles and the puck are passive pieces of plastic. The game therefore
necessitates a heavy and expensive air table to generate a thin film of air,
which reduces the friction between the puck and the table.

The premise of this project is to design an actuator similar to an air hockey
puck, which can generate its own air film and thus function on any flat surface.

Talk about:
Cost

Scope of project (i.e. not going to actually build one, calculate with
approximations, then simulate)

Clarify approach (steps)

\asterbreak

However, not everything is fully functional, and many areas could be improved.
Some of these, mostly the hardware changes, are described in section
project, a lot of time was spent researching solutions. In addition to this, the
hardware design required more time than expected.
energy consumption results in a substantial drop in power usage. It can also
help to make statistics and comparisons between devices and households.

\begin{description}
  \item[Consumption] : indicate where energy is being consumed and by whom, as
    well as extract statistics, which can be linked to other factors, like for
    instance thermostat settings, outdoor temperature, activity in the
    home\ldots
  \item[React] to the environment : turn lights and other appliances on and off
    depending on activity and time of day. It would be interesting for example
    to turn off lights when no human presence is detected, to use power-hungry
    devices during the night to even out power usage, or to adjust the lighting
    to suit the amount of natural light coming in a room. 
\end{description}

The benefits of such a system are twofold : first, it has been
demonstrated\cite{darby2006} that simply providing people with data about their
energy consumption results in a substantial drop in power usage. It can also
help to make statistics and comparisons between devices and households.

\section{Conclusion}


The hardware design is functional, and can be demonstrated as requested.
However, not everything is fully functional, and many areas could be improved.
Some of these, mostly the hardware changes, are described in section
project, a lot of time was spent researching solutions. In addition to this, the
hardware design required more time than expected.

All these, I believe, could be worth exploring in the future, and I hope that
this platform may serve a purpose beyond the scope of this semester project.

\vspace{3cm}
Ecublens, le xx Juin 2011

\vspace{2cm}
Andrew Watson



\pagebreak
%%%%%%%%%%%%%%%%%%%%%%%%%%%%%%%%%%%%%%%%%%%%%%%%%%%%%%%%%%%%%%%%%%%%%%%%%%%%%%%
\bibliographystyle{ieeetr}
\bibliography{biblio-airhockey}

\pagebreak
%%%%%%%%%%%%%%%%%%%%%%%%%%%%%%%%%%%%%%%%%%%%%%%%%%%%%%%%%%%%%%%%%%%%%%%%%%%%%%%
\appendix

\section{Design changes}
\label{sec:design-changes}
\end{document}

