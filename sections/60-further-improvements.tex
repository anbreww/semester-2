% TODO : some introduction sentence

The first downside of the proposed design is that we are using harmonics that
are way too high. As a result, it will be difficult to excite exactly the right
mode, and we risk having too much parasitic movement which will impede the
efficiency of the device.

The problem is that the membrane is not sufficiently rigid, and therefore the
base frequency is too low. There are two ways to alleviate this problem:

\begin{itemize}
  \item Use a smaller membrane: the smaller the membrane is, the higher the
    frequency will be. Current simulations suggest that it should be possible to
    shrink the membrane and still generate enough force to lift the device off
    the table.
  \item Use more rigid materials: the problem with this is that the system is
    limited by its most supple element, the plastic puck. In order to use a more
    rigid membrane, we would have to use a different material for the puck
    itself, or rigidify the setup somehow, e.g. by adding a layer of aluminium
    underneath the puck.
\end{itemize}

The second downside of this design is that it is not cost-effective. In the
beginning, the goal of the project was to define a topology that allowed the use
of a standard actuator, which would be easier and cheaper to source than a
custom-made actuator. Unfortunately, this proved to be unsuccessful, and we are
now left with a solution which diverged from the original plan, and uses a type
of actuator that seems quite difficult and expensive to obtain, mainly due to
its large size. Here are two possible solutions:

\begin{itemize}
  \item Use a design with a bending ring. The original design proposed by M.
    Wiesendanger uses a bending ring, and seems to function quite well. This
    project intentionally explored other solutions rather than going with the
    known working solution. 
  \item Rather than use one big ring, place several smaller actuators at various
    positions around the membrane. This would complicate the simulations
    somewhat, and require that the actuator be simulated in three dimensions.
    Since time was limited and the simulation software unfamiliar to the author,
    this was not attempted within this project.
\end{itemize}

\asterbreak

It is unfortunate that all the time spent on this project was put into
simulations and calculations, without leading to a functional prototype. This
would have been the ideal way to verify the simulations, if a practical model
had been obtained through these methods. Instead, we have explored several
options and topologies through simulations, which allowed us to discard some
options, but has not really converged towards an ideal implementable solution. 

The most likely solution now seems to be the one which was intentionally put to
the side in the beginning, namely the design using a bending ring.



