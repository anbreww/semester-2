Before we start with simulations, we shall verify that the specifications are
attainable with a few analytical approximations. By using some simplified
formulas and approximations, we shall calculate the membrane movement and
diameter needed to support the device.

\subsection{Membrane movement}
\label{sub:displacement}

In our first calculations, we will assume that the movement is identical across
the whole membrane. This means that the behaviour at the boundary is the same as
in the center of the disc. This is obviously far from true in our case, but it
should be sufficient to gain an understanding of the orders of magnitude
involved, and whether or not the design is feasible.

The complete demonstration can be found in Chapter 5 of Markus Wiesendanger's
PhD Thesis \cite{wiesendanger2001} and will therefore not be repeated here as it
is beyond the scope of this project.

We will note however that the pressure $P_\infty$ inside an air bearing can be approximated
by the following equation, where $\Psi_\infty$ is a measure of mass per unit
area, taken from the Reynods equation.

\begin{equation}
  \bar{P}_\infty = {{\Psi_\infty} \over {\sqrt{1 - \varepsilon}}} =
  {{P_B^2 (1 + \frac{3}{2} \varepsilon_B^2)} \over {\sqrt{1 - \varepsilon}}}
  \label{eq:pinf-epsilon}
\end{equation}

\Eq{eq:pinf-epsilon} gives the pressure locally generated by a relative
displacement $\varepsilon_B$ at the boundary, given boundary pressure $P_B$. If
we substitute ambient pressure for $P_B$ and plot the values of $P_\infty$ for
relative displacements of 5\% to 20\%, we get the following graph, where a value
of 1 corresponds to ambient pressure:

\begin{figure}[h]
  \begin{center}
    \includegraphics[width=0.5\textwidth]{calcs/pinf-epsilon}
  \end{center}
  \caption{Pressure ($P_\infty$) rises relative to displacement ($\varepsilon$)}
  \label{fig:pinf-epsilon}
\end{figure}

As shown in \FIG{fig:pinf-epsilon}, a relative displacement $\varepsilon =
0.20$ (which corresponds to 20\% of the base airgap $h_0$) yields a local
overpressure of approximately 8\% of ambient.

\subsection{Air pressure vs membrane diameter}
\label{sub:pressure-diameter}

Now that we have established how much pressure we can create with a given
movement of the membrane, we need to calculate the minimum actuator size which
will produce a sufficient force to lift the whole device.

Let us call $\bar{P}_N$ the pressure necessary to lift the device. This is given
relative to ambient pressure $P_0$, the vertical force exercted by gravity, and
the surface upon which the pressure is exerted.

\begin{equation}
  \bar{P}_N = \frac{m g}{\pi r^2 p_a} + P_0,
  \quad \quad P_0 = \frac{p_0}{p_a} = 1
  \label{eq:pinf-mg}
\end{equation}

As expected, plotting the required pressure with regard to the membrane radius
gives a curve where the pressure decreases quadratically as the radius
increases. This is shown in \FIG{fig:pinf-epsilon}.

\begin{figure}[h]
  \begin{center}
    \includegraphics[width=0.5\textwidth]{calcs/pinf-r}
  \end{center}
  \caption{Pressure ($P_N$) necessary to lift the device decreases as radius
  ($r$) increases}
  \label{fig:pinf-epsilon}
\end{figure}

\subsection{Displacement vs diameter}
\label{sub:displacement-diameter}

If we now wish to combine these two values, we must ensure that the pressure
generated by the displacement of the membrane is at least as high as what is
needed to lift the device. This translates to :

\begin{equation}
  \bar{P}_\infty \geq \bar{P}_N
  \label{eq:pinfty}
\end{equation}

And therefore

\begin{equation}
  \bar{P}_\infty - \bar{P}_N \geq 0
  \label{eq:pinfty0}
\end{equation}

We can plot \Eq{eq:pinfty0} against the radius (shown in metres) and the
relative displacement :

\begin{figure}[h]
  \begin{center}
    \includegraphics[width=0.7\textwidth]{calcs/p-p}
  \end{center}
  \caption{Pressure difference given by radius [m] and relative displacement}
  \label{fig:p-p}
\end{figure}

\Fig{fig:p-p} shows the resulting surface given by \EQ{eq:pinfty0}. The pressure
difference $\bar{P}_\infty - \bar{P}_N$ is represented on the vertical axis. The
flat surface exists to show the zero plane. Wherever the curved surface is above
the zero plane, we have a positive over-pressure, which should result in the
device lifting off the ground.

From this we can deduce that our best bet is to use a membrane as large as
possible, while remaining smaller than the outer diameter of the puck, so as to
avoid damage in case of collisions. With a diameter of \SI{60}{\milli\metre} for
instance, a relative displacement of just over 5\% should provide sufficient
lift for the system.

To be conservative, we shall aim for a relative displacement of 10\% which,
given an airgap of \SI{10}{\micro\metre} corresponds to a membrane movement of
\SI{1}{\micro\metre}. % TODO : add chosen membrane diameter

\subsection{Lever}
\label{sub:lever}

From what has been shown up until now, it seems that the movement of the
actuator itself will not be sufficient for the device to levitate. To do so
would require either a very large stack, which would make the device
inconveniently thick, or a very high supply voltage, which will make the design
of the power stage unnecessarily complicated.

Therefore, we will attempt to amplify the mechanical movements by other means.
We have already seen the possibility of using a bender ring actuator
\cite{wiesendanger2001}. Here, we would like to see if it is possible to amplify
the movement of the actuator using a lever mechanism, by using the actuator to
push the membrane around a pivot point.

Assuming this is possible (if the pivot point does not absorb all the energy),
we can approximate the behaviour in the following way:

\begin{figure}[h]
  \begin{center}
    \includegraphics[width=0.9\textwidth]{lever}
  \end{center}
  \caption{Lever geometry}
  \label{fig:lever}
\end{figure}

As shown in \FIG{fig:lever}, the membrane is of radius $R_2$. The support is
placed at a distance $r_1$ from the center of the actuator.  The displacement of
the actuator, $\delta_a$, yields a displacement $\delta_e$ at the end of the
membrane, if it behaves as expected. To simplify the equation, we use $r_1'$ to
represent the distance between the support and the edge of the actuator.

From the schematic drawing, we can deduce the following equation to link
actuator displacement to the displacement at the end of the membrane:

\begin{equation}
  \delta_e = \delta_a \cdot \frac{R_2-r_1}{r_1'}
  \label{eq:displacement}
\end{equation}

Using a \SI{4}{\milli\metre} stack, we calculate the displacement according to
the specifications given by Noliac\furl{http://www.noliac.com} :

\begin{equation}
  \delta_a = \SI{2.6}{\micro\metre} \cdot \frac{U}{\SI{200}{\volt}}
  \label{eq:delta-a}
\end{equation}

For an operating voltage of \SI{20}{\volt}, this gives us a movement of
$\delta_a = \SI{260}{\nano\metre}$. We would like movement in the order of
\SI{3}{\micro\metre}. Therefore, we need :

\begin{equation}
  \frac{R_2 - r_1}{r_1'} = \frac{\delta_e}{\delta_a} = 
  {\SI{3}{\micro\metre} \over \SI{0.26}{\micro\metre}} = 11.5
  \label{eq:11-5}
\end{equation}

This means that in theory, the part of the membrane which overhangs from the
supports must be almost twelve times longer than the distance between the
actuator and the support. Given the width of the support and the size of the
device, this pretty much excludes this type of topology.
