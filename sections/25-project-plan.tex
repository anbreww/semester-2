The project is divided into several stages.

The first stage involves getting familiar with the existing literature on the
subject of air bearings and film damping.

From here, it was possible to use some available analytical models, and through
some simplification, to evaluate which parameters will be important in the
design of the actuator and membrane, as well as estimate whether the desired
topology seems feasible or not given the environmental properties.

% TODO : rewrite that last sentence

Once it has been established which parameters play a role in the design of the
system, a simplified version of it will be analyzed in the frequency domain
using finite element analysis. This should allow us to chart the resonant
frequency according to various parameters. Since we are restricted to a certain
range of frequencies, this will allow us to make a first selection among the
various topologies proposed.

Once we have confined our choice to a small subset of possible topologies, we
can add some complexity to the models that we are simulating. The first step
here is to move from frequency domain simulations to time domain simulations, as
the first only give an idea of the shape of the movement, without giving any
numerical values of displacement. This is due to the fact that the
eigenfrequency simulation aims to find the exact resonant frequencies of a
model, and without any damping, the movement is in theory infinite in this case.

By simulating in the time domain, we can compare various topologies together,
and verify whether the choices we made with regard to operating voltages and
actuator size / layers are sane. In part %TODO : find reference
, we will have calculated a minimum displacement necessary to lift the system.
We will want to make sure that the results of the time domain simulation are
coherent with the results from this section.

Once we have obtained one or more topologies that fit our requirements, we can
add the film damping element to our simulation. By using several of COMSOL's
multiphysics modules, we can combine several types of interaction in one single
model. Here, we will compute the displacement of the membrane when it is excited
at its resonant frequency. The membrane displacement is then re-used as an input
to the film damping module, which will calculate the pressure over the surface
of the membrane.

Once we have this information, we can average the pressure to obtain a vertical
force, and observe the evolution of this force over time. If all goes well,
after a few excitation cycles, this force should become sufficient to lift the
system off the ground.

In all these simulations, we assume that the mean airgap is constant. In
reality, it will be affected by the performance of the system. If the system
creates more pressure than necessary, the gap will increase. Since the absolute
displacement of the membrane is constant, the relative displacement of the
membrane with regard to the mean thickness will therefore decrease. The system
should therefore be expected to stabilise at a certain height, which will depend
on all the parameters seen previously.

In the final round of simulations, we would like to take into account the weight
of the system and gravitational force to verify these assumptions.
