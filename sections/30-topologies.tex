In the primary stages of the projects, several topologies were proposed for
study. The aim here is to begin with a list of possible designs, which we can
then analyze further, rejecting the designs which seem too complex or
inefficient, to end up with one or more ideal topologies.

These topologies can be divided into several categories. Note that all these
illustrations, even though they lack symmetry axes, are actually cut views of an
axisymmetric design. Supports or actuators that appear rectangular are therefore
to be understood as discs.

Topologies \numrange{1}{3} were proposed from the beginning of the project.
After finding that neither of these gave suitable results, some modifications
were proposed, in the form of topologies 4 and 5. The final results of this
project are shown for topology 4.

\subsection{Topology 1}

The first \fig{fig:topo-4} is the most basic design. It is comprised of the plastic air hockey
puck (yellow), the piezo-electric actuator (brown) and the membrane (white).

\begin{figure}[h]
  \begin{center}
    \includegraphics[width=0.7\textwidth]{topo-4.png}
  \end{center}
  \caption{Simple membrane with actuator (topo 1)}
  \label{fig:topo-4}
\end{figure}

This design is mostly shown for the sake of completeness, and because it is the
simplest possible to build. However, since the goal of the project is to create
a puck which is suitable for playing on a table, it will have to be reasonably
resistant. In this design, the membrane is glued to the actuator, which is glued
to the ABS support.

The actuator itself is quite fragile, and especially sensitive to impact. Since
this design does not provide any additional supports to the membrane, it will
not be explored further. However, if the goal was to make a stationary device,
or one that would not be subjected to external impact forces, this design might
be worth analyzing.

\subsection{Topology 2}

Topology number two is a variation on the first proposal. Here, we add a
supporting ring around the actuator, which will serve two functions.

\begin{figure}[h]
  \begin{center}
    \includegraphics[width=0.7\textwidth]{topo-2.png}
  \end{center}
  \caption{Membrane with supports (topo 2)}
  \label{fig:topo-2}
\end{figure}

The first function is simply to protect the actuator, and provide a strong
mechanical link between the plastic support and the membrane. The actuator
itself cannot be expected to withstand strong shear forces. The support ring,
which will probably be machined out of aluminium and glued to both parts of the
device, should be able to absorb external forces that the device would endure.

The second function, which will be described more in detail in section
\tref{sub:lever} is to provide a mechanical lever. We expect that if the support
is close to the actuator, it will act as a pivot point for the membrane, and the
movement will be amplified as we go further away from the actuator.

\begin{figure}[h]
  \begin{center}
    \includegraphics[width=0.7\textwidth]{topo-1.png}
  \end{center}
  \caption{Membrane with supports on the outside (topo 2b)}
  \label{fig:topo-1}
\end{figure}

\Fig{fig:topo-1} is a variation of this design, where the supports have been
moved to the outside of the membrane. While it does not warrant a section of its
own since it is simply a special case of this topology, it bears mentioning that
it has some particularities which set it apart from the rest.

Notably, the lever effect which we expect from this design will be absent from
this case. We could however expect the membrane to bellow between the support
and the actuator, and so still provide a higher maximum displacement in some
areas than right above the actuator.

Furthermore, we expect this design to be one of the most mechanically robust,
since the outside of the membrane will be entirely connected to the plastic
support. It does not offer any loose ends that could break upon impact with the
edge of the playing surface, for instance.


\subsection{Topology 3}

The third design is similar to the previous in that the membrane is also
supported by a metallic ring. However, in this case the actuator is no longer a
stack connected between the membrane and the plastic puck. This time, we use a
bender ring type, which is glued entirely to the membrane.

\begin{figure}[h]
  \begin{center}
    \includegraphics[width=0.7\textwidth]{topo-3.png}
  \end{center}
  \caption{Bender ring setup (topo 3)}
  \label{fig:topo-3}
\end{figure}

The main advantage of this design is that the excursion of the membrane is no
longer limited to the displacement of the actuator itself. It can move freely
around the support ring, being supported only in one point. Another advantage
that may come from this design is that in case of shocks, there should be little
damage on the actuator : it is thinner than a piezo stack, and therefore has
less inertia (which would provoke shear forces along the seam); in addition to
this, the fact that it is only connected to one of the components of the puck
means that it will not sustain any damage if the membrane moves sideways
relative to the puck.

%%% TODO : explain why we're not going to use this topo

\subsection{Topology 4}

The topologies shown previously were discarded for reasons which will be
explicited later in this report. As a consequence, two new topologies were
proposed. Topology 4 will be the focus of the final part of the project, as it
seems to give the best results.

% TODO : make a schematic for TOPO 4
\begin{figure}[h]
  \begin{center}
    \includegraphics[width=0.7\textwidth]{topo-5.png}
  \end{center}
  \caption{Using a ring actuator (topo 4)}
  \label{fig:topo-5}
\end{figure}

It is built on the same principle as Topology 2b, with a slight difference :
instead of having the actuator in the centre of the device, it has now been
moved to the outside, close to the supports. While this seems like a fairly
trivial modification on paper, this means in reality that the actuator is now a
large ring, which is much more costly to manufacture than the standard stack
which was used in the previous designs.

Hopefully, this design will provide sufficient lift so that the diameter of the
membrane can be decreased, and thus use a smaller actuator.


\subsection{Topology 5}

The final proposition came up towards the end of the project, and therefore has
not been simulated. It was discovered during the tests with the aforementioned
toplogies that the plastic support was not sufficiently rigid to allow the
device to function at a high frequency : beyond a certain frequency, the puck
would vibrate more than the membrane. This design is meant to address this
issue.

\begin{figure}[h]
  \begin{center}
    \includegraphics[width=0.7\textwidth]{topo-5.png}
  \end{center}
  \caption{Using the device upside down : membrane as a counterweight (topo 5)}
  \label{fig:topo-5}
\end{figure}

Instead of vibrating a membrane close to the table, this design uses the
opposite approach : the plastic disc is machined to offer a flat surface, and is
used to generate the airgap. The part which previously acted as the membrane is
now a counterweight, fitted on the top of the device. Since the plastic support
is only a fraction of the total weight of the device, it is expected to move up
and down faster than the device as a whole. Rather than articulating it around
a set of supports, it relies on its inertia to provide a stable base for the
pumping motion.
