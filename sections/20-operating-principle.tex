% TODO : explain principle

The self-levitating puck does not need an external air supply. Instead, it
creates its own air film by vibrating at a high frequency.

The puck is composed of four main elements:

% TODO : create picture with letters
\begin{figure}[h]
  \begin{center}
    \includegraphics[width=0.8\textwidth]{overview}
  \end{center}
  \caption{Overall view}
  \label{fig:overview}
\end{figure}

% TODO : convert to list with letters
% TODO : make this list more brief, try to make a mory lengthy, wordy and easy
% to follow explanation
\begin{enumerate}[A]
  \item The original puck, made of yellow ABS plastic. This is the base around
    which the system is built. The main idea of this project is to re-use a
    standard puck, according to USAA rules. With the addition of the other
    components, it will no longer be acceptable by official game rules though.
  \item The membrane, which is made of FR-4 (an epoxy mainly used in circuit
    board design). This membrane will resonate to create the pressure necessary
    to lift the contraption.
  \item An aluminium support ring will provide mechanical robustness to the
    device, as well as provide one fixed point for the membrane to move around.
  \item A piezo-electric actuator, in this case a ring stack, will actuate
    the membrane, exciting it at one of its resonant frequencies, in order to
    get the best possible vertical displacement.
\end{enumerate}


\subsection{Squeeze film}

% TODO : explain how the squeeze film works. Why there is more air coming in
% than going out. add graphs from Wiesendanger

\begin{figure}[h]
  \begin{center}
    \includegraphics[width=0.7\textwidth]{boundary-airflow}
  \end{center}
  \caption{Boundary airflow}
  \label{fig:boundary-airflow}
\end{figure}

\begin{figure}[h]
  \begin{center}
    \includegraphics[width=0.7\textwidth]{nonlinear-p-h-relationship}
  \end{center}
  \caption{Non-linear relationship between pressure and air film thickness}
  \label{fig:nonlinear-relationship}
\end{figure}
