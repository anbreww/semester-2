The self-levitating puck does not need an external air supply. Instead, it
creates its own air film by vibrating at a high frequency. This behaviour is
that of a squeeze film, which will be described in \tref{sub:squeezefilm}.

The puck is composed of four main elements:

\begin{figure}[h]
  \begin{center}
    \includegraphics[width=0.8\textwidth]{overview}
  \end{center}
  \caption{Overall view}
  \label{fig:overview}
\end{figure}

\begin{enumerate}[A]
  \item The original puck, made of yellow ABS plastic. This is the base around
    which the system is constructed. The main idea of this project is to re-use a
    standard puck (as defined by USAA\footnote{United States Air-Table-Hockey
    Association} rules). With the addition of the other
    components, it will no longer be acceptable by official game rules though.
  \item The membrane, which is made of FR-4 (an epoxy mainly used in circuit
    board design). This membrane will resonate to create the pressure necessary
    to lift the contraption.
  \item An aluminium support ring will provide mechanical robustness to the
    device, as well as provide one fixed point for the membrane to move around.
  \item A piezo-electric actuator, in this case a ring stack, will actuate
    the membrane, exciting it at one of its resonant frequencies, in order to
    get the best possible vertical displacement.
\end{enumerate}

\subsection{Actuator}

The designs proposed in this report all use a piezoceramic actuator.
Piezoelectric materials convert an electric field to a mechanical movement, and
vice versa. Given an applied voltage $V$ and an actuator thickness $h$, the
field is given by

\begin{equation}
  E = \frac{V}{h}
  \label{eq:field}
\end{equation}

The deformation along each axis of the piezo element is given by its
coefficients. In the axis of polarization, this results in a deformation

\begin{equation}
  \Delta h = V d_{33}
  \label{eq:delta-h}
\end{equation}

The piezoceramic chosen from Noliac has a value of
\SI{330e-12}{\coulomb\per\newton}. The deformation obtained
under an applied voltage of \SI{200}{\volt} is approximately
\SI{66}{\nano\metre}. This is much too small for the application we wish to use
it for. This is why we will be using a multilayer piezo stack, which interleaves
layers of piezo ceramics with electrodes, allowing the electric field to be
multiplied by the number of layers, while keeping the applied voltage relatively
low. 

In addition to this, we will be actuating the system at its resonant frequency,
in order to gain some mechanical amplification.

\Fig{fig:force-stroke} shows an example of a force to stroke diagram for a
piezoelectric actuator. The position of the free stroke and blocking force
depend on the voltage : the higher the voltage, the larger the area under the
curve, up to a maximum tolerated voltage, which is usually around
\SI{200}{\volt}, at least for the actuators considered in this project.

A piezoelectric actuator offers a compromise between force and movement. The
further it moves, the less force it can apply. It is important to keep this in
mind when selecting an actuator, as it will not be able to supply both its
maximum rated force nor its maximum stroke, both at the same time. The ideal
area to operate is at half the maximum value, where work:

\begin{equation}
  \mathrm{Work = Force \cdot Displacement} = 
  \frac{\textrm{free stroke}}{2} \cdot \frac{\textrm{blocking force}}{2}
  \label{eq:work}
\end{equation}

is at its maximum.

\begin{figure}[h!]
  \begin{center}
    \includegraphics[width=0.7\textwidth]{force-stroke}
  \end{center}
  \caption{Relationship between stroke and force for a given voltage}
  \label{fig:force-stroke}
\end{figure}

\subsection{Squeeze film}
\label{sub:squeezefilm}

The puck operates according to the principle of a squeeze film. In this
situation, an object can be lifted by a thin film of air, which is generated by
the object itself. The puck is considered to be stationary, and set at a certain
distance $h_0$ above the playing surface (the table). This distance is of the
order of \SI{10}{\micro\metre}.

The membrane is made to oscillate at a given frequency (typically around
\SIrange{20}{40}{\kilo\hertz}). Since the object has a certain inertia, it will
remain almost immobile compared to the membrane. Consequentially, the thickness
of the air film trapped between the membrane and the table will vary
periodically with the movement of the membrane. The air film thus gets
alternatively compressed and decompressed, creating forces that pull the puck
towards the table, then push it away. 

The pressure created by this back and forth movement is a non-linear function of
the airgap thickness. \Fig{fig:nonlinear-relationship} illustrates the
relationship between airgap $h$ and pressure $p$. In the case of air under
standard conditions, it behaves according to the following equation, with $n
\approx 1.4$

\begin{equation}
  pV^{n} \sim ph^{n} \approx const
  \label{eqn:pv-ph}
\end{equation}

In \FIG{fig:nonlinear-relationship}, the airgap is assumed to be varying
harmonically around a mean thickness $h_0$. Due to the non-linear relationship
between the airgap and the pressure, the mean pressure is actually slightly
higher than ambient pressure. In \cite{wiesendanger2001}, the airgap is given as

\begin{equation}
  h(t) = h_0 (1 - \varepsilon cos(\omega t)), \quad 0 \leq \varepsilon \leq 1
  \label{eqn:airgap-periodic}
\end{equation}

and therefore the medium pressure can be obtained by integrating the pressure
over one cycle:

\begin{equation}
  \bar{p} = 
  \frac{p_0 h_0}{2\pi} \int_{0}^{2\pi}\!\frac{1}{h(t)^n} \partial \omega t = 
  \frac{p_0}{\sqrt{1-\varepsilon^2}}
  \label{eqn:pmean}
\end{equation}

\Eq{eqn:pmean} gives us an approximation of the medium pressure under ideal
conditions (we have so far neglected boundary conditions). We can use this
result later to calculate how much relative variation we will need to provide
sufficient pressure to lift the puck. Here, $\varepsilon$ is the relative
displacement with regard to the mean airgap.

\begin{figure}[h]
  \begin{center}
    \includegraphics[width=0.7\textwidth]{nonlinear-p-h-relationship}
  \end{center}
  \caption{Non-linear relationship between pressure and air film
  thickness\cite{wiesendanger2001}}
  \label{fig:nonlinear-relationship}
\end{figure}


The previous calculations hold true if there is no airflow at the boudaries of
the system. However, in the system that will be studied in this project, air can
escape the system all around the membrane. 

Luckily, the flow of air is proportional to $h^3$, the third power of the
airgap, and the pressure gradient $\partial p / \partial x$. When the air is
being compressed, the pressure gradient will become larger, and tend to push air
out of the system. Since the airgap is elevated to the third power, at this
point $h$ is becoming smaller, and therefore $h^3$ is becoming very small.
Therefore, very little air escapes the system.

\begin{figure}[h]
  \begin{center}
    \includegraphics[width=0.7\textwidth]{boundary-airflow}
  \end{center}
  \caption{Boundary airflow\cite{wiesendanger2001}}
  \label{fig:boundary-airflow}
\end{figure}

On the other hand, when the membrane is getting farther away, the pressure is
lowering proportionally to the distance, whereas $h^3$ is increasing at a far
greater rate. As a result, the system will absorb more air during the
decompression cycle than during the compression cycle, and will always maintain
a sufficient amount of air to create an overpressure.

\Fig{fig:boundary-airflow} illustrates the various components that influence the
flow of air at the boundaries. The full description and demonstration can be
found in \cite{wiesendanger2001}.


