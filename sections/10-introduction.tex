The game of air hockey is played with a puck and two mallets. The pucks are
thin slabs of plastic which ``float'' on a table, such as can be seen on this
report's title page. The table is comprised of a smooth playing surface,
surrounded by a rail. The playing surface is drilled with many small holes,
which allow air generated by the table itself to seep through, and form a
cushion upon which the puck can glide.

The mallets on the other hand are backed with felt, and thus do not need to
float on the air cusion. Players hit the air hockey puck with the mallet to try
to send it into the opponent's goals, which are situated on either end of the
table.

\begin{figure}[h]
  \begin{center}
    \includegraphics[width=0.5\textwidth]{mallets}
  \end{center}
  \caption{Standard air hockey pucks and mallets}
  \label{fig:mallets}
\end{figure}

The problem that is seen with this approach is that the game necessitates heavy
and expensive tables, since they need to incorporate all the machinery necessary
to create the air cushion.

This projects aims to determine whether it might be feasible to move the
function of generating an air cushion from the table to the puck itself, by
using a piezo-electric actuator and a flexible membrane to create an
overpressure between the puck and a flat playing surface.

This would allow the game to be easily transported and played anywhere, as long
as a flat surface of sufficient dimensions is available. In addition to this, it
will probably reduce costs, as the components required to assemble a
self-levitating air hockey puck will very likely be much cheaper than the
machinery of an air table.

\asterbreak

Even though the aim of this project is to provide an actuator for use in the
game of air hockey, it remains more of a proof of concept. Since the puck is
likely to withstand very high impact force from the mallet, a design using a
piezo-electric actuator glued to the main support element is not very suitable.
The actuator itself will probably break very early, if it does not simply detach
one way or another.

Nevertheless, it is an interesting topic to study, and self-levitating actuators
certainly do have applications at the moment, namely in air bearings, where it
can be difficult to inject air from an external source.

This report will walk through the steps taken to simulate the behaviour of a
hypothetical air hockey puck, with the aim of assembling a prototype of the
final design.
