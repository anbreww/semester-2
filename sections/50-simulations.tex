\section{Simulations}

The analytical approximations presented in the previous section tend to show
that the specifications we defined are suitable for the design of an actuator.
We can now move onto more complex models, using finite element analysis. 

All the results presented in this section were obtained using COMSOL 3.5a. All
simulations were made using an axi-symmetric model, which should produce a fair
approximation of the desired result. This was done mainly to simplify things and
reduce calculation times. Once a suitable model is obtained though, it would be
worth running a complete simulation in three dimensions to account for the fact
that the membrane does not move exactly the same way all around its axis of
revolution.

\subsection{Eigenfrequency approach}

In the first round of simulations, we used COMSOL's Piezo Axial Symmetry module
to calculate the eigenfrequency of various topologies, and understand how the
system deforms at various frequencies. This step should allow us to select a few
suitable designs to analyze further in the time domain, which takes more time to
simulate.

To do this, we chose a basic topology, and varied the main parameters (namely
thickness of the membrane and dimensions of the supporting ring) and took note
of the fundamental resonant frequency, as well as the first harmonic.

The objective throughout this was to design the system with a standard actuator
in mind, so as to reduce cost of parts, and make it easier to source parts in
case a prototype was to be constructed.

\subsubsection{Analytical model}

The results in this section are not totally relevant to the final design, as the
materials were not correctly defined when these simulations were made. The
results referenced in the table below are quite suitable to create a model, but
they are not applicable to the final design. The reason for this is that the
materials that we wish to use are less rigid than the ones used in this
simulation. As a result, the actual base frequencies are much too low to be of
use (since we are aiming to operate above \SI{20}{\kilo\hertz}).

However, they will still be shown, to illustrate the path that was taken to get
to the end result, as well as to serve as a possible path for future tests.

The model used was based on Topology 2, using a support diameter of
\SI{67}{\milli\metre}, with a membrane \SI{50}{\milli\metre} in diameter. 

To obtain the results given in \TAB{tab:eigenfrequencies}, we use 

\begin{table}[h]
  \centering
    \begin{tabular}{l|l|l|l|l|l|l}
      r[mm]  & \multicolumn{2}{c|}{\SI{0.8}{\milli\metre}} &
		 \multicolumn{2}{c|}{\SI{1.6}{\milli\metre}} & 
		 \multicolumn{2}{c}{\SI{2.4}{\milli\metre}} \\
        \hline
        5  & 1779 & 6560 & 3217 & 6767  & 3997 & 7679  \\ 
        6  & 1950 & 6624 & 3478 & 6861  & 4238 & 7834  \\ 
        7  & 2116 & 6529 & 3770 & 6996  & 4517 & 8057  \\ 
        8  & 2393 & 6739 & 4104 & 7063  & 4799 & 8302  \\ 
        9  & 2681 & 6790 & 4480 & 7224  & 5099 & 8613  \\ 
        10 & 3036 & 6826 & 4868 & 7435  & 5410 & 8880  \\ 
        11 & 3456 & 7058 & 5323 & 7662  & 5731 & 9538  \\ 
        12 & 3981 & 7106 & 5784 & 8242  & 5948 & 10481 \\ 
        13 & 4627 & 7284 & 6163 & 8908  & 6386 & 11284 \\ 
        14 & 5417 & 7469 & 6552 & 9915  & 6808 & 12042 \\ 
        15 & 6231 & 7939 & 6862 & 11183 & 7008 & 13658 \\ 
        16 & 6850 & 9021 & 7066 & 12877 & 7242 & 15044 \\
    \end{tabular}
    \caption{Eigenfrequency for various positions and membrane thicknesses.}
    \label{tab:eigenfrequencies}
\end{table}

The size of the mesh is important to obtain repeatable results. The parameters
used in these simulations are referenced in \TAB{tab:mesh}. For the model used,
the maximum element size can be varied between approximately 0.00010 and 0.00030
and produce usable results. Any finer, and the simulation time becomes very
long, any wider and the results begin to vary too much. For some of the higher
frequencies, a mesh element size of 0.00010 had to be used to obtain repeatable
results.  No area-specific parameters were used. The actuator is loaded with a
voltage difference of \SI{200}{\volt}.

\begin{table}[h]
  \centering
  \begin{tabular}{l|l}
    Maximum Element Size		&	0.00015	\\
    Maximum Element size scaling factor	&	0.25	\\
    Element growth rate			&	1.2	\\
    Mesh curvature factor		&	0.25	\\
    Mesh curvature cutoff		&	0.0003	\\
    Resulution of narrow margins	&	1	\\
    Refinement method			&	Regular	\\
  \end{tabular}
  \caption{Mesh parameters}
  \label{tab:mesh}
\end{table}

Plotting the results of \TAB{tab:eigenfrequencies} produces the curves seen in
\FIG{fig:frequencies}. In this figure, the top three curves represent the first
harmonic, whereas the bottom three represent the base frequency, for each
different actuator thickness. 

\begin{figure}[h]
  \begin{center}
    \includegraphics[width=0.7\textwidth]{frequencies}
  \end{center}
  \caption{Plot of eigenfrequencies.}
  \label{fig:frequencies}
\end{figure}

The system behaves as expected. If we assimilate the membrane to a cantilever
beam, we can assume that it behaves similarly. In the case of a beam, the
resonant frequency is given by the following equation \cite{roark2002}:

\begin{equation}
  f = \frac{K_n}{2\pi} \sqrt{\frac{E I}{w L^4}}
  \label{eq:beam}
\end{equation}

Where $K_n = 3.52$ for the first vibration mode, $E$ is the Young's modulus. The
lever is subjected to a uniform load $w$ and $L$ is the free unsupported length.
If we assimilate $I$ to a square section, we have $I_x = {b h^3}/{12}$.

Therefore, we can assume that the frequency is proportional to $\sqrt{h^3}$ and
$\sqrt{L^4} = L^2$, where L is the unsupported length (overhanging from the
support ring).

\subsubsection{Problems with this approach}

% TODO : explain what the problem is

Since the previous results were inconclusive, the final time domain simulations
were made with a different topology, as well as using a higher frequency. Since
the results were no longer very consistent when using higher order harmonics,
the approach presented here was abandoned, and instead, the topology to study
was chosen empirically according to the shape of the membrane deformations.

For the actuator to work efficiently, we wanted the following characteristics :
\begin{enumerate}
  \item Lowest possible harmonic, since the further we go, the more difficult it
    is to excite only one vibration mode.
  \item A reasonable number of ripples.
  \item Not have an inflexion point at the point of contact with the actuator.
\end{enumerate}

The last item in particular is important. The greatest movement will happen at
the antinodes. If the actuator is placed on a node, it will not be able to
excite the membrane at its resonant frequency. Instead, we will want to place it
somewhere in between, so that it can exicte the membrane, but still leave some
room for mechanical amplification.

\subsection{Time domain simulation}

Once a topology has been chosen in the frequency domain, it is time to run it
through a time domain simulation, which takes more time.

In this stage, we use the same COMSOL module as before, and add an external
voltage on the boundaries of the piezo actuator. This time, instead of applying
a fixed voltage, we actuate it with a sinusoidal function, as the frequency that
was calculated in the previous simulation. For a given voltage $V$ and frequency
$f$ we give the following electric potential:

\begin{equation}
  v_0 = V sin(2\pi f t)
  \label{eq:v0}
\end{equation}

With this, we can simulate over a few periods, and verify that the displacement
seems sufficient, according to our specifications and estimations made in
\tref{sec:calculations}. After a few iterations of running a time simulation,
modifying the topology, calculating the eigenfrequency, and running a new
simulation at the new frequency, we arrived at a satisfactory topology.

\begin{figure}[h]
  \begin{center}
    \includegraphics[width=0.7\textwidth]{final-topo}
  \end{center}
  \caption{Final topology : support ring is at the far right, actuator left.}
  \label{fig:final-overview}
\end{figure}

\subsubsection{Quality factor}
\subsubsection{Adding film damping}
