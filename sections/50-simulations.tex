\section{Simulations}

The analytical approximations presented in the previous section tend to show
that the specifications we defined are suitable for the design of an actuator.
We can now move onto more complex models, using finite element analysis. 

All the results presented in this section were obtained using COMSOL 3.5a. All
simulations were made using an axi-symmetric model, which should produce a fair
approximation of the desired result. This was done mainly to simplify things and
reduce calculation times. Once a suitable model is obtained though, it would be
worth running a complete simulation in three dimensions to account for the fact
that the membrane does not move exactly the same way all around its axis of
revolution.

\subsection{Eigenfrequency approach}

In the first round of simulations, we used COMSOL's Piezo Axial Symmetry module
to calculate the eigenfrequency of various topologies, and understand how the
system deforms at various frequencies. This step should allow us to select a few
suitable designs to analyze further in the time domain, which takes more time to
simulate.

To do this, we chose a basic topology, and varied the main parameters (namely
thickness of the membrane and dimensions of the supporting ring) and took note
of the fundamental resonant frequency, as well as the first harmonic.

The objective throughout this was to design the system with a standard actuator
in mind, so as to reduce cost of parts, and make it easier to source parts in
case a prototype was to be constructed.

\subsubsection{Analytical model}

The results in this section are not totally relevant to the final design, as the
materials were not correctly defined when these simulations were made. The
results referenced in the table below are quite suitable to create a model, but
they are not applicable to the final design. The reason for this is that the
materials that we wish to use are less rigid than the ones used in this
simulation. As a result, the actual base frequencies are much too low to be of
use (since we are aiming to operate above \SI{20}{\kilo\hertz}).

However, they will still be shown, to illustrate the path that was taken to get
to the end result, as well as to serve as a possible path for future tests.

The model used was based on Topology 2, using a support diameter of
\SI{67}{\milli\metre}, with a membrane \SI{50}{\milli\metre} in diameter. 

By running simulations for different values of support position ($r$) and
membrane thickness ($e$), we obtain a series of resonant frequencies, which can
be found in \TAB{tab:eigenfrequencies}.

\begin{table}[h]
  \centering
    \begin{tabular}{l|l|l|l|l|l|l}
      r[mm]  & \multicolumn{2}{c|}{\SI{0.8}{\milli\metre}} &
		 \multicolumn{2}{c|}{\SI{1.6}{\milli\metre}} & 
		 \multicolumn{2}{c}{\SI{2.4}{\milli\metre}} \\
        \hline
        5  & 1779 & 6560 & 3217 & 6767  & 3997 & 7679  \\ 
        6  & 1950 & 6624 & 3478 & 6861  & 4238 & 7834  \\ 
        7  & 2116 & 6529 & 3770 & 6996  & 4517 & 8057  \\ 
        8  & 2393 & 6739 & 4104 & 7063  & 4799 & 8302  \\ 
        9  & 2681 & 6790 & 4480 & 7224  & 5099 & 8613  \\ 
        10 & 3036 & 6826 & 4868 & 7435  & 5410 & 8880  \\ 
        11 & 3456 & 7058 & 5323 & 7662  & 5731 & 9538  \\ 
        12 & 3981 & 7106 & 5784 & 8242  & 5948 & 10481 \\ 
        13 & 4627 & 7284 & 6163 & 8908  & 6386 & 11284 \\ 
        14 & 5417 & 7469 & 6552 & 9915  & 6808 & 12042 \\ 
        15 & 6231 & 7939 & 6862 & 11183 & 7008 & 13658 \\ 
        16 & 6850 & 9021 & 7066 & 12877 & 7242 & 15044 \\
    \end{tabular}
    \caption{Eigenfrequency for various positions and membrane thicknesses.}
    \label{tab:eigenfrequencies}
\end{table}

The size of the mesh is important to obtain repeatable results. The parameters
used in these simulations are referenced in \TAB{tab:mesh}. For the model used,
the maximum element size can be varied between approximately 0.00010 and 0.00030
and produce usable results. Any finer, and the simulation time becomes very
long, any wider and the results begin to vary too much. For some of the higher
frequencies, a mesh element size of 0.00010 had to be used to obtain repeatable
results.  No area-specific parameters were used. The actuator is loaded with a
voltage difference of \SI{200}{\volt}.

\begin{table}[h]
  \centering
  \begin{tabular}{l|l}
    Maximum Element Size		&	0.00015	\\
    Maximum Element size scaling factor	&	0.25	\\
    Element growth rate			&	1.2	\\
    Mesh curvature factor		&	0.25	\\
    Mesh curvature cutoff		&	0.0003	\\
    Resulution of narrow margins	&	1	\\
    Refinement method			&	Regular	\\
  \end{tabular}
  \caption{Mesh parameters}
  \label{tab:mesh}
\end{table}

Plotting the results of \TAB{tab:eigenfrequencies} produces the curves seen in
\FIG{fig:frequencies}. In this figure, the top three curves represent the first
harmonic, whereas the bottom three represent the base frequency, for each
different actuator thickness. 

\begin{figure}[h]
  \begin{center}
    \includegraphics[width=0.7\textwidth]{frequencies}
  \end{center}
  \caption{Plot of eigenfrequencies.}
  \label{fig:frequencies}
\end{figure}

The system behaves as expected. If we assimilate the membrane to a cantilever
beam, we can assume that it behaves similarly. In the case of a beam, the
resonant frequency is given by the following equation \cite{roark2002}:

\begin{equation}
  f = \frac{K_n}{2\pi} \sqrt{\frac{E I}{w L^4}}
  \label{eq:beam}
\end{equation}

Where $K_n = 3.52$ for the first vibration mode, $E$ is the Young's modulus. The
lever is subjected to a uniform load $w$ and $L$ is the free unsupported length.
If we assimilate $I$ to a square section, we have $I_x = {b h^3}/{12}$.

Therefore, we can assume that the frequency is proportional to $\sqrt{h^3}$ and
$\sqrt{L^4} = L^2$, where L is the unsupported length (overhanging from the
support ring).

\subsubsection{Problems with this approach}

As explained at the beginning of this section, these results are unfortunately
not usable in the final design, since the materials were not defined correctly
in the simulation. However, as we will see later, the last simulations seem to
indicate that the membrane is more than large enough to provide sufficient lift
to the system. Therefore, it is probably possible to reduce its size while
keeping sufficient vertical force, thus raising the resonant frequency. If the
frequency rises sufficiently, then this approach can indeed be used.

Since the previous results were inconclusive, the final time domain simulations
were made with a different topology, as well as using a higher frequency. Since
the results were no longer very consistent when using higher order harmonics,
the approach presented here was abandoned, and instead, the topology to study
was chosen empirically according to the shape of the membrane deformations.

For the actuator to work efficiently, we wanted the following characteristics :
\begin{enumerate}
  \item Lowest possible harmonic, since the further we go, the more difficult it
    is to excite only one vibration mode.
  \item A reasonable number of ripples.
  \item Not have an inflexion point at the point of contact with the actuator.
\end{enumerate}

The last item in particular is important. The greatest movement will happen at
the antinodes. If the actuator is placed on a node, it will not be able to
excite the membrane at its resonant frequency. Instead, we will want to place it
somewhere in between, so that it can exicte the membrane, but still leave some
room for mechanical amplification.

\subsection{Time domain simulation}
\label{sub:time-domain}

Once a topology has been chosen in the frequency domain, it is time to run it
through a time domain simulation, which takes more time.

In this stage, we use the same COMSOL module as before, and add an external
voltage on the boundaries of the piezo actuator. This time, instead of applying
a fixed voltage, we actuate it with a sinusoidal function, as the frequency that
was calculated in the previous simulation. For a given voltage $V$ and frequency
$f$ we give the following electric potential:

\begin{equation}
  v_0 = V sin(2\pi f t)
  \label{eq:v0}
\end{equation}

With this, we can simulate over a few periods, and verify that the displacement
seems sufficient, according to our specifications and estimations made in
\tref{sec:calculations}. After a few iterations of running a time simulation,
modifying the topology, calculating the eigenfrequency, and running a new
simulation at the new frequency, we arrived at a satisfactory topology.


With this topology, we obtain a resonant frequency $f = \SI{24055}{\hertz}$.
When the simulation is run in the frequency domain, it is not possible to
calculate values for the deflection, as it mainly solves for the eigenfrequency,
where the displacement is theoretically infinite. The time domain simulation
answers this by simulating over a period of time, and taking into account
damping within the materials.

The final topology used in these simulations can be found in Appendix
\ref{sec:simulation-results}.

%\pagebreak
\subsubsection{Simulation parameters}
In the simulations made from \ref{sub:time-domain}, the following parameters
were used.

The support is made of a yellow ABS plastic.

\begin{table}[!h]
  \centering
  \begin{tabular}{l|l|l}
    \multicolumn{3}{l}{Structural settings (Decoupled, Isotropic)}	\\
    \hline
    $E$		& \SI{2e9}{\pascal} 	& Young's Modulus 	\\
    $v$		& \SI{0.33}{}		& Poisson's ratio	\\
    $\rho$	& \SI{1150}{\kilo\gram\per\cubic\metre}	& Density	\\
  \end{tabular}
  \caption{Plastic support}
  \label{tab:properties-plastic}
\end{table}

The choice of materials for the membrane was open. We decided to opt for FR-4,
which is usually employed for the fabrication of printed circuit boards, as it
is easy to obtain in various thicknesses as well as easy to machine.

\begin{table}[!h]
  \centering
  \begin{tabular}{l|l|l}
    \multicolumn{3}{l}{Structural settings (Decoupled, Isotropic)}	\\
    \hline
    $E$		& \SI{400e6}{\pascal} 	& Young's Modulus 	\\
    $v$		& \SI{0.33}{}		& Poisson's ratio	\\
    $\rho$	& \SI{1850}{\kilo\gram\per\cubic\metre}	& Density	\\
  \end{tabular}
  \caption{FR4 Membrane}
  \label{tab:properties-membrane}
\end{table}


The parameters for the piezo actuator were taken from the manufacturer's
datasheet. The material chosen was Noliac's NCE46\furl{
http://www.noliac.com/Files/Billeder/02\%20%
Standard/Ceramics/Noliac_CEramics_NCE_datasheet.pdf }, since it offered the best
specifications within the ``Hard'' piezo category, as well as a high quality
factor ($Q_m$), which will lead to less energy dissipation within the actuator
itself.

\begin{table}[!h]
  \centering
  \begin{tabular}{l|l|l}
    \multicolumn{3}{l}{Structural settings (Piezoelectric)}	\\
    \hline
    \multicolumn{2}{l|}{Material model}	& Piezoelectric		\\
    \multicolumn{2}{l|}{Constitutive form}	& Stress-charge form		\\
    \multicolumn{2}{l|}{Material orientation}	& xz plane		\\
    \hline
    $E$		& \SI{400e6}{\pascal} 	& Young's Modulus 	\\
    $v$		& \SI{0.33}{}		& Poisson's ratio	\\
    $\rho$	& \SI{7700}{\kilo\gram\per\cubic\metre}	& Density	\\
  \end{tabular}
  \caption{FR4 Membrane}
  \label{tab:properties-membrane}
\end{table}

To take into account the energy absorbed by the materials themselves, the
following parameters were added under the ``Damping'' tab:

\begin{table}[!h]
  \centering
  \begin{tabular}{l|l|l}
    \multicolumn{3}{l}{Structural Damping : Rayleigh}	\\
    \hline
    $\alpha_{dM}$ & \SI{11}{\per\second} & Mass damping parameter	\\
    $\beta_{dK}$  & \SI{0}{\second} & Stiffness damping parameter	\\
  \end{tabular}
  \caption{Damping settings}
  \label{tab:properties-damping}
\end{table}

\subsubsection{Adding film damping}

Now that we have established that the membrane displacement is sufficient, we
can combine the film damping module with the piezo module to calculate the
pressure along the surface of the membrane. To do this, we keep the same
parameters as before in the piezo module, and add the displacement variable to
the film damping module. The specific parameters for the film damping simulation
can be found in \FIG{fig:film-damping-settings}

\begin{figure}[!h]
  \begin{center}
    \includegraphics[width=0.6\textwidth]{film-damping-settings}
  \end{center}
  \caption{Film damping parameters}
  \label{fig:film-damping-settings}
\end{figure}

The simulation is ran for time values between 0 and \SI{2}{\milli\second}.
Unfortunately, the simulation does not quite stabilize after that time period,
but the simulation length is limited by the quantity of memory on the computer
running the simulations.

Given the above input, the film damping module returns a pressure profile along
the membrane, which is represented graphically in \FIG{fig:pressure-profile}.
Larger images, with legends can be found in appendix
\ref{sec:simulation-results}.

\begin{figure}[h]
  \begin{center}
    \includegraphics[width=0.7\textwidth]{topo-f-24055-film-crop}
  \end{center}
  \caption{Graphical representation of the pressure profile along the membrane}
  \label{fig:pressure-profile}
\end{figure}

To know if this is sufficient to lift the membrane, we must integrate the
pressure over the whole surface of the membrane. COMSOL provides a sinmple way
of integrating the quantity over the surface. This gives us a force for each
time step that we have analyzed.

We can then repeat the calculation automatically for each time step, which gives
us the following graph for one period:

\begin{figure}[h]
  \begin{center}
    \includegraphics[width=0.7\textwidth]{one-period}
  \end{center}
  \caption{Vertical force for one actuator period}
  \label{fig:one-period}
\end{figure}

As can be seen in \FIG{fig:one-period}, the sine-like function is slightly
assymetrical around zero. The average force is therefore positive (upwards).
In order to quantify the force, we must calculate the average over one period. 

\begin{equation}
  F = \frac{1}{T} \int_{0}^{T} f(t) dt
  \label{eq:force}
\end{equation}

The result of \EQ{eq:force} was calculated by exporting the time data to MATLAB,
and using the \bash{trapz} function to perform a numerical integration. For the
time period shown in \FIG{fig:one-period}, the result is a mean upward force of
\SI{1460}{\newton}.  
This is clearly much too high, and is probably due to the fact that the vertical
position of the membrane relative to the table is fixed in the simulation. The
evolution of the vertical force over the duration of the simulation is
referenced in \FIG{fig:full-run}.

\begin{figure}[h]
  \begin{center}
    \includegraphics[width=0.7\textwidth]{full-run}
  \end{center}
  \caption{Plot of vertical force over the whole \SI{2}{\milli\second}
  simulation}
  \label{fig:full-run}
\end{figure}

Unfortunately, there was not enough time to push the simulations any further.
Some improvements over the current status will be proposed in
\tref{sec:other-solutions} : \nameref{sec:other-solutions}
